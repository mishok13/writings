\documentclass{beamer}

\usepackage{mathtools}

\begin{document}

\title{How To Build Tile Server Using OpenStreetMap tools and data}
\author{Andrii Mishkovskyi}
\date{March 2011}

\maketitle

\begin{frame}{Topics}
  \begin{itemize}
  \item GIS basics
  \item OpenStreetMap data
  \item Rendering with Mapnik
  \end{itemize}
\end{frame}

\section{GIS basics}

\begin{frame}{What is GIS}
  Geographical information system
  Collect, analyze and manage data that can be linked to location.
  Collect hence cartography
  Analyze statistical analysis
  Manage hence database
  GIS stands somewhere between cartography, statistical analysis and software engineering
\end{frame}

\begin{frame}{Collecting data}
  GPS devices
  Aerial photography
  Survey
  etc
\end{frame}

\begin{frame}{Representing data}
  Raster images -- heatmaps, staticmaps
  Vector data
  Textual (geocoding)
\end{frame}

\begin{frame}{Formats of data}
  WKB
  WKT
  GeoJSON for unstructured data
  GML
  KML
\end{frame}

\begin{frame}{Ways to store data}
  PostGIS
  MySQL Spatial
  Oracle Spatial
  MS SQL
\end{frame}

\begin{frame}{Processing data in Python}
  GDAL
  Proj.4
  Shapely
  OGR
\end{frame}


\section{OpenStreetMap}

What is openstreetmap?
What kind of data openstreetmap provides?
What form



\section{Accessing tiles}

Accessing schema
Slippy map, TMS, WMS

\section{Rendering}

Mapnik
Style file, layers and datasources
Fonts
Symbolizers


\end{document}
