\documentclass{beamer}

\usepackage{amsmath}
\usepackage{mathtools}
\usepackage{listings}
\usepackage{graphicx}

\lstset{
  basicstyle=\footnotesize\ttfamily
}

\begin{document}

\title{How To Build Tile Server Using OpenStreetMap tools and data}
\author{Andrii Mishkovskyi}
\date{March 2011}

\maketitle

\begin{frame}
  \frametitle{Outline}
  \tableofcontents
\end{frame}


\section{Objectives and prerequisites}

\begin{frame}{What will you learn?}
  \begin{itemize}
  \item Cartography basics
  \item How to render maps with Mapnik
  \item How to build basic map service with Flask
  \end{itemize}
\end{frame}

\begin{frame}{First of all}
  \begin{itemize}
  \item Grab the CD or flash with all the data from me
  \item Use materials at content.mishkovskyi.net/pycon2011
  \item Or use handouts
  \end{itemize}
\end{frame}

\section{Introduction}

\subsection{Terms, definitions and classification}

\begin{frame}
  \frametitle{Terms}
  \begin{description}
  \item[Map] Graphic representation of the geographical setting
  \item[Cartography] Science and practice of making maps
  \item[GIS] A system that collects, analyzes, manages or processes in any other way data linked to location.
  \end{description}
\end{frame}

\begin{frame}
  \frametitle{Applications of GIS}
  \begin{itemize}
  \item Geography
  \item Cartography
  \item Navigation
  \item Search engines
  \item Remote sensing, land surveying, urban planning \ldots{}
  \end{itemize}
\end{frame}

\begin{frame}
  \frametitle{Characteristics of maps}
  \begin{itemize}
  \item<1-> Reduction \uncover<2->{\(\implies\) Scale}
  \item<3-> Transformation \uncover<4->{\(\implies\) Map projection}
  \item<5-> Abstractions \uncover<4->{\(\implies\) Symbolism}
  \end{itemize}
\end{frame}

\begin{frame}
  \frametitle{Classification of maps}
  \begin{itemize}
  \item By scale -- small-scale world, large-scale your neighborhood
  \item By function -- general reference, thematic, charts
  \item By subject matter -- cadastre, plans
  \end{itemize}
\end{frame}


\subsection{Projection}

\begin{frame}
  \frametitle{Projection}
  Conversion of spherical information into 2d pane
\end{frame}

\begin{frame}
  \frametitle{Characteristics of projection}
  \begin{itemize}
  \item Scale factor
  \item Area distortion
  \item Angle distortion
  \end{itemize}
\end{frame}




\subsection{Geodata: storing, managing and processing }

\begin{frame}
  \frametitle{Classes of data}
  \begin{itemize}
  \item Physical (Topographical -- elevations, terrain \& water objects)
  \item Cultural
  \item Human-made
  \end{itemize}
\end{frame}

\begin{frame}
  \frametitle{Collecting data}
  \begin{itemize}
  \item Surveying
  \item GPS tracks
  \item Tracing aerial imagery
  \end{itemize}
\end{frame}

\begin{frame}
  \frametitle{Sources of data}
  \begin{itemize}
  \item Mapping companies (TeleAtlas, Navteq, Ordnance Survey)
  \item Public sources (OpenStreetMap, Natural Earth)
  \item Government institutions (NASA, local government)
  \item Personal data (GPS tracks, social networks)
  \end{itemize}
\end{frame}

\begin{frame}
  \frametitle{Storing data}
  \begin{itemize}
  \item RDBMS (PostgreSQL + PostGIS, MySQL Spatial, Oracle Spatial, Spatialite)
  \item Non-relational databases (Neo4j, MongoDB, CouchDB)
  \item Flat files (WKB, WKT, GeoJSON, KML, GML)
  \end{itemize}
\end{frame}

\begin{frame}
  \frametitle{Simple Feature acess: SQL}
  \begin{itemize}
  \item Full set of querying functions
  \item Tons of useful features for on-the-fly projecting etc.
  \item Almost fully supported by PostGIS
  \end{itemize}
\end{frame}

\begin{frame}
  \frametitle{Other standards}
  \begin{itemize}
  \item Well-known text
  \item Well-known binary
  \item Geography Markup Language
  \item Keyhole Markup Language
  \item Shapefiles
  \end{itemize}
\end{frame}

\begin{frame}
  \frametitle{How OSM handles data}
  Special XML format used in
  \begin{itemize}
  \item Official API
  \item Official plant extracts
  \item Unofficial eXtended API (XAPI)
  \end{itemize}
\end{frame}

\subsection{OSM data}

\begin{frame}[fragile]
  \frametitle{How OSM XML looks like}
  \lstinputlisting[language=XML]{osm-xml-overview.xml}
\end{frame}

\begin{frame}
  \frametitle{Shortly describing each of those}
  \begin{description}
  \item[Node] Point
  \item[Way] Roads
  \item[Relation] Collection of different geometry features
  \end{description}
\end{frame}

\begin{frame}
  \frametitle{Nodes}
  \lstinputlisting[language=XML]{osm-nodes-overview.xml}
\end{frame}

\begin{frame}
  \frametitle{Ways}
  \lstinputlisting[language=XML]{osm-ways-overview.xml}
\end{frame}

\begin{frame}
  \frametitle{Relations}
  \lstinputlisting[language=XML]{osm-relations-overview.xml}
\end{frame}

\begin{frame}
  \frametitle{Parsing OSM data}
  \begin{itemize}
  \item Osmosis
  \item osm2pgsql
  \item tons of little scripts
  \end{itemize}
\end{frame}

\begin{frame}
  \frametitle{osm2pgsql database}
  Imports to PostgreSQL with the following tables
  \begin{itemize}
  \item point
  \item line
  \item polygon
  \item nodes
  \end{itemize}
\end{frame}

\begin{frame}
  \frametitle{Fetching data from PostgreSQL}
  \lstinputlisting{simple-sql-query.sql}[language=SQL]
\end{frame}

\begin{frame}
  \frametitle{Limiting search by bounding box}
  \lstinputlisting{sql-query-with-bbox.sql}[language=SQL]
\end{frame}

\begin{frame}
  \frametitle{More silly examples}
  \lstinputlisting{sql-examples.sql}[language=SQL]
\end{frame}

\section{Mapnik}

\begin{frame}
  \frametitle{What is Mapnik}
  Rendering library for
\end{frame}

\begin{frame}
  \frametitle{Mapnik concepts}
  \begin{itemize}
  \item Style
  \item Map
  \item Rule
  \item Layer
  \item Datasource
  \end{itemize}
\end{frame}

\subsection{Pure Python rendering}

\begin{frame}
  \frametitle{Rendering map}

\end{frame}


\begin{frame}
  \frametitle{Creating style}

\end{frame}

\begin{frame}
  \frametitle{Creating layer}

\end{frame}

\begin{frame}
  \frametitle{Linking layers and styles}

\end{frame}

\begin{frame}
  \frametitle{Selecting area to render}

\end{frame}

\subsection{Simplifying matters with XML}

\begin{frame}
  \frametitle{Overview of map file in XML}

\end{frame}

\begin{frame}
  \frametitle{Defining styles in XML}

\end{frame}

\begin{frame}
  \frametitle{Defining rules in XML}

\end{frame}

\begin{frame}
  \frametitle{Defining layers in XML}

\end{frame}

\subsection{Real world example}

\begin{frame}
  \frametitle{Fonts and fontsets}

\end{frame}

\begin{frame}
  \frametitle{Including PostGIS datasources}

\end{frame}

\begin{frame}
  \frametitle{Changing level of details aka zoom levels}

\end{frame}

\begin{frame}
  \frametitle{Adding roads}

\end{frame}


\begin{frame}
  \frametitle{Adding POIs}

\end{frame}


\section{Imagery server}

\begin{frame}
  \frametitle{API}

\end{frame}

\begin{frame}[fragile]
  \frametitle{Hello world API}
  \ttfamily{http://localhost:5000/}
  \begin{centering}
    \includegraphics[scale=1.0]{simple-map.png}
  \end{centering}
\end{frame}



% Let's see our first example
% ...
% Looks scary, but do not fear, we'll go through every part of this nice and easy

% First of all, let's look at the line with the projection. It conforms to the
% previously discussed way of producing projection descriptions used by Proj.4

% TODO: actual descriptions prior to this

% Then, there's a simple Map creation mechanism. Map contains so called layers,
% which define the way the image layers are stacked and styles, which are named
% collections of rules, which describe the way geographical features are going
% to be represented

% Layers

% In classic cartography layers represent the actual physical layer. That is,
% layers correspond to the topographic view

% Styles

% Styles are set of rules

% Rules

% They define the way geographical features get rendered in the image, within
% the given scale.




% Mapnik
% Style file, layers and datasources
% Fonts
% Symbolizers

% To tile server

% WMS
% Tile API
% Different ways
% Google, Bing, Mapquest, Yahoo
% Building tile server from scratch
% Staticmaps

% Flask + Mapnik

% Changing styles on the fly
% Staticmaps
% Get tiles in bounding box


% http://msdn.microsoft.com/en-us/library/bb259689.aspx

\end{document}
