\documentclass{beamer}

\usepackage{mathtools}

\begin{document}

\title{How To Build Tile Server Using OpenStreetMap tools and data}
\author{Andrii Mishkovskyi}
\date{March 2011}

\maketitle

\begin{frame}{What will you learn?}
  \begin{itemize}
  \item Cartography basics
  \item How to render cartography imagery with Mapnik
  \item How to build basic cartography web service with Werkzeug
  \end{itemize}
\end{frame}

\begin{frame}{First of all}
  \begin{itemize}
  \item Grab the CD or flash with all the data from me
  \item Copy the content somewhere
  \item Get prepared
  \end{itemize}
\end{frame}

\section{Introduction}

\subsection{GIS basics}

\begin{frame}{What is GIS}
  Geographical information system
  Collect, analyze and manage data that can be linked to location.
  Collect hence cartography
  Analyze statistical analysis
  Manage hence database
  GIS stands somewhere between cartography, statistical analysis and software engineering
\end{frame}

\begin{frame}{Collecting data}
  GPS devices
  Aerial photography
  Survey
  etc
\end{frame}

\begin{frame}{Representing data}
  Raster images -- heatmaps, staticmaps
  Vector data
  Textual (geocoding)
\end{frame}

\begin{frame}{Formats of data}
  \begin{itemize}
  \item Well known binary/text
  \item GeoJSON for unstructured data
  \item GML
  \item KML
  \end{itemize}
\end{frame}

\begin{frame}{Ways to store data}
  PostGIS
  MySQL Spatial
  Oracle Spatial
  MS SQL
\end{frame}

\begin{frame}{Processing data in Python}
  GDAL
  Proj.4
  Shapely
  OGR
\end{frame}

\subsection{OpenStreetMap}

\begin{frame}{Data providers}
  Data doesn't come from nowhere
  Local governments, non-profit organizations, for-profit organizations
Crowd-sourcing
\end{frame}

\begin{frame}{Crowd sourcing?}
  Paid-for data can get expensive really fast
  Non-profit organizations are the thing of the past
  Government is rarely interested in high quality
\end{frame}

\begin{frame}{OpenStreetMap}
  CC-by-SA license, switching to ODBL soon
  350k users
  Data quality greatly varies (depends on community)
\end{frame}

\begin{frame}{Data structure}
highly unstructured
lat,lon,tags
\end{frame}

\begin{frame}{Tag system}

\end{frame}

\begin{frame}{OSM API}
  \begin{itemize}
  \item basic structutre
  \item transactional
  \item requests
  \end{itemize}
\end{frame}

\begin{frame}{OSM XAPI}
  \begin{itemize}
  \item Unofficial
  \item Large requests
  \item Recommended
  \end{itemize}
\end{frame}

\begin{frame}{Downloads}
  \begin{itemize}
  \item Planet.osm (whole world)
  \item regions osm files (CloudMade, GeoFabrik)
  \end{itemize}
\end{frame}

\begin{frame}{Handling data}
  \begin{itemize}
  \item osm2pgsql
  \item Osmosis
  \item Write your own
  \end{itemize}
\end{frame}

\section{Rendering}



\end{frame}


\section{HTTP access}

\begin{frame}{Access types}
  WMS
  WMTS
  Short description follows
\end{frame}

\begin{frame}{WMS}

\end{frame}

Let's see our first example
...
Looks scary, but do not fear, we'll go through every part of this nice and easy

First of all, let's look at the line with the projection. It conforms to the
previously discussed way of producing projection descriptions used by Proj.4

TODO: actual descriptions prior to this

Then, there's a simple Map creation mechanism. Map contains so called layers,
which define the way the image layers are stacked and styles, which are named
collections of rules, which describe the way geographical features are going
to be represented

Layers

In classic cartography layers represent the actual physical layer. That is,
layers correspond to the topographic view

Styles

Styles are set of rules

Rules

They define the way geographical features get rendered in the image, within
the given scale.




Mapnik
Style file, layers and datasources
Fonts
Symbolizers

To tile server

WMS
Tile API
Different ways
Google, Bing, Mapquest, Yahoo
Building tile server from scratch
Staticmaps

Flask + Mapnik

Changing styles on the fly
Staticmaps
Get tiles in bounding box


http://msdn.microsoft.com/en-us/library/bb259689.aspx

\end{document}
