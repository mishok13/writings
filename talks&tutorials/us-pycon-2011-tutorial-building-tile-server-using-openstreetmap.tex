\documentclass{beamer}

\usepackage{mathtools}

\begin{document}

\title{How To Build Tile Server Using OpenStreetMap tools and data}
\author{Andrii Mishkovskyi}
\date{March 2011}

\maketitle

\begin{frame}{Topics}
  \begin{itemize}
  \item GIS basics
  \item OpenStreetMap data
  \item Rendering with Mapnik
  \end{itemize}
\end{frame}

\section{Introduction}

\subsection{GIS basics}

\begin{frame}{What is GIS}
  Geographical information system
  Collect, analyze and manage data that can be linked to location.
  Collect hence cartography
  Analyze statistical analysis
  Manage hence database
  GIS stands somewhere between cartography, statistical analysis and software engineering
\end{frame}

\begin{frame}{Collecting data}
  GPS devices
  Aerial photography
  Survey
  etc
\end{frame}

\begin{frame}{Representing data}
  Raster images -- heatmaps, staticmaps
  Vector data
  Textual (geocoding)
\end{frame}

\begin{frame}{Formats of data}
  WKB
  WKT
  GeoJSON for unstructured data
  GML
  KML
\end{frame}

\begin{frame}{Ways to store data}
  PostGIS
  MySQL Spatial
  Oracle Spatial
  MS SQL
\end{frame}

\begin{frame}{Processing data in Python}
  GDAL
  Proj.4
  Shapely
  OGR
\end{frame}


\subsection{OpenStreetMap}

\begin{frame}{Data providers}
  Data doesn't come from nowhere
  Local governments, non-profit organizations, for-profit organizations
Crowd-sourcing
\end{frame}

\begin{frame}{Crowd sourcing?}
  Paid-for data can get expensive really fast
  Non-profit organizations are the thing of the past
  Government is rarely interested in high quality
\end{frame}

\begin{frame}{OpenStreetMap}
  CC-by-SA license, switching to ODBL soon
  350k users
  Data quality greatly varies (depends on community)
\end{frame}

\section{Rendering}

\subsection{How-to}

\begin{frame}{}
  Get the data
  Put the geometric features in the predefined way

\end{frame}

\subsection{Mapnik}

\begin{frame}{}

\end{frame}


\section{HTTP access}

\begin{frame}{Access types}
  WMS
  WMTS
  Short description follows
\end{frame}

\begin{frame}{WMS}

\end{frame}

Let's see our first example
...
Looks scary, but do not fear, we'll go through every part of this nice and easy

First of all, let's look at the line with the projection. It conforms to the
previously discussed way of producing projection descriptions used by Proj.4

TODO: actual descriptions prior to this

Then, there's a simple Map creation



Mapnik
Style file, layers and datasources
Fonts
Symbolizers


\end{document}
