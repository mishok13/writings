\documentclass{beamer}

\usepackage{amsmath}
\usepackage{mathtools}

\begin{document}

\title{How To Build Tile Server Using OpenStreetMap tools and data}
\author{Andrii Mishkovskyi}
\date{March 2011}

\maketitle

\section{Objectives and prerequisites}

\begin{frame}{What will you learn?}
  \begin{itemize}
  \item Cartography basics
  \item How to render maps with Mapnik
  \item How to build basic map service with Flask
  \end{itemize}
\end{frame}

\begin{frame}{First of all}
  \begin{itemize}
  \item Grab the CD or flash with all the data from me
  \item Use materials at content.mishkovskyi.net/pycon2011
  \item Or use handouts
  \end{itemize}
\end{frame}

\section{Introduction}

\subsection{Terms, definitions and classification}

\begin{frame}
  \frametitle{Terms}
  \begin{description}
  \item[Map] Graphic representation of the geographical setting
  \item[Cartography] Science and practice of making maps
  \item[GIS] A system that collects, analyzes, manages or processes in any other way data linked to location.
  \end{description}
\end{frame}

\begin{frame}
  \frametitle{Applications of GIS}
  \begin{itemize}
  \item Geography
  \item Cartography
  \item Navigation
  \item Search engines
  \item Remote sensing, land surveying, urban planning \ldots{}
  \end{itemize}
\end{frame}

\begin{frame}
  \frametitle{Characteristics of maps}
  \begin{itemize}
  \item<1-> Reduction \uncover<2->{\(\implies\) Scale}
  \item<3-> Transformation \uncover<4->{\(\implies\) Map projection}
  \item<5-> Abstractions \uncover<4->{\(\implies\) Symbolism}
  \end{itemize}
\end{frame}

\begin{frame}
  \frametitle{Classification of maps}
  \begin{itemize}
  \item By scale -- small-scale world, large-scale your neighborhood
  \item By function -- general reference, thematic, charts
  \item By subject matter -- cadastre, plans
  \end{itemize}
\end{frame}

  % Geographical information system
  % Collect, analyze and manage data that can be linked to location.
  % Collect hence cartography
  % Analyze statistical analysis
  % Manage hence database
  % GIS stands somewhere between cartography, statistical analysis and software engineering

\subsection{Geodata: storing, managing and processing }

\begin{frame}
  \frametitle{Classes of data}
  \begin{itemize}
  \item Physical (Topographical -- elevations, terrain \& water objects)
  \item Cultural
  \item Human-made
  \end{itemize}
\end{frame}

\begin{frame}
  \frametitle{Collecting data}
  \begin{itemize}
  \item Surveying
  \item GPS tracks
  \item Tracing aerial imagery
  \end{itemize}
\end{frame}

\begin{frame}
  \frametitle{Sources of data}
  \begin{itemize}
  \item Mapping companies (TeleAtlas, Navteq, Ordnance Survey)
  \item Public sources (OpenStreetMap, Natural Earth)
  \item Government institutions (NASA, local government)
  \item Personal data (GPS tracks, social networks)
  \end{itemize}
\end{frame}

\begin{frame}
  \frametitle{Storing data}
  \begin{itemize}
  \item RDBMS (PostgreSQL + PostGIS, MySQL Spatial, Oracle Spatial, Spatialite)
  \item Non-relational databases (Neo4j, MongoDB, CouchDB)
  \item Flat files (WKB, WKT, GeoJSON, KML, GML)
  \end{itemize}
\end{frame}

\begin{frame}
  \frametitle{Simple Feature acess: SQL}
  \begin{itemize}
  \item Full set of querying functions
  \item Tons of useful features for on-the-fly projecting etc.
  \item Almost fully supported by PostGIS
  \end{itemize}
\end{frame}

\begin{frame}
  \frametitle{Other standards}
  \begin{itemize}
  \item Well-known text
  \item Well-known binary
  \item Geography Markup Language
  \item Keyhole Markup Language
  \end{itemize}
\end{frame}

\begin{frame}{Representing data}
  Raster images -- heatmaps, staticmaps
  Vector data
  Textual (geocoding)
\end{frame}

\begin{frame}{Formats of data}
  \begin{itemize}
  \item Well known binary/text
  \item GeoJSON for unstructured data
  \item GML
  \item KML
  \end{itemize}
\end{frame}

\begin{frame}{Ways to store data}
  PostGIS
  MySQL Spatial
  Oracle Spatial
  MS SQL
\end{frame}

\begin{frame}{Processing data in Python}
  GDAL
  Proj.4
  Shapely
  OGR
\end{frame}

\subsection{OpenStreetMap}

\begin{frame}{Data providers}
  Data doesn't come from nowhere
  Local governments, non-profit organizations, for-profit organizations
Crowd-sourcing
\end{frame}

\begin{frame}{Crowd sourcing?}
  Paid-for data can get expensive really fast
  Non-profit organizations are the thing of the past
  Government is rarely interested in high quality
\end{frame}

\begin{frame}{OpenStreetMap}
  CC-by-SA license, switching to ODBL soon
  350k users
  Data quality greatly varies (depends on community)
\end{frame}

\begin{frame}{Data structure}
highly unstructured
lat,lon,tags
\end{frame}

\begin{frame}{Tag system}

\end{frame}

\begin{frame}{OSM API}
  \begin{itemize}
  \item basic structutre
  \item transactional
  \item requests
  \end{itemize}
\end{frame}

\begin{frame}{OSM XAPI}
  \begin{itemize}
  \item Unofficial
  \item Large requests
  \item Recommended
  \end{itemize}
\end{frame}

\begin{frame}{Downloads}
  \begin{itemize}
  \item Planet.osm (whole world)
  \item regions osm files (CloudMade, GeoFabrik)
  \end{itemize}
\end{frame}

\begin{frame}{Handling data}
  \begin{itemize}
  \item osm2pgsql
  \item Osmosis
  \item Write your own
  \end{itemize}
\end{frame}

\section{Rendering}




\section{HTTP access}

\begin{frame}{Access types}
  WMS
  WMTS
  Short description follows
\end{frame}

\begin{frame}{WMS}

\end{frame}

Let's see our first example
...
Looks scary, but do not fear, we'll go through every part of this nice and easy

First of all, let's look at the line with the projection. It conforms to the
previously discussed way of producing projection descriptions used by Proj.4

TODO: actual descriptions prior to this

Then, there's a simple Map creation mechanism. Map contains so called layers,
which define the way the image layers are stacked and styles, which are named
collections of rules, which describe the way geographical features are going
to be represented

Layers

In classic cartography layers represent the actual physical layer. That is,
layers correspond to the topographic view

Styles

Styles are set of rules

Rules

They define the way geographical features get rendered in the image, within
the given scale.




Mapnik
Style file, layers and datasources
Fonts
Symbolizers

To tile server

WMS
Tile API
Different ways
Google, Bing, Mapquest, Yahoo
Building tile server from scratch
Staticmaps

Flask + Mapnik

Changing styles on the fly
Staticmaps
Get tiles in bounding box


http://msdn.microsoft.com/en-us/library/bb259689.aspx

\end{document}
