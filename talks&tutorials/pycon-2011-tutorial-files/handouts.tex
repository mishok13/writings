\documentclass{article}

\usepackage{listings}
\usepackage{color}
\usepackage{courier}

\lstset{ %
% language=Octave,                % choose the language of the code
basicstyle=\ttfamily,       % the size of the fonts that are used for the code
numbers=left,                   % where to put the line-numbers
numberstyle=\footnotesize,      % the size of the fonts that are used for the line-numbers
stepnumber=1,                   % the step between two line-numbers. If it's 1 each line
                                % will be numbered
numbersep=5pt,                  % how far the line-numbers are from the code
backgroundcolor=\color{white},  % choose the background color. You must add \usepackage{color}
showspaces=false,               % show spaces adding particular underscores
showstringspaces=false,         % underline spaces within strings
showtabs=false,                 % show tabs within strings adding particular underscores
frame=single,	                % adds a frame around the code
tabsize=2,	                % sets default tabsize to 2 spaces
captionpos=b,                   % sets the caption-position to bottom
breaklines=true,                % sets automatic line breaking
breakatwhitespace=false,        % sets if automatic breaks should only happen at whitespace
title=\lstname,                 % show the filename of files included with \lstinputlisting;
                                % also try caption instead of title
escapeinside={\%*}{*)},         % if you want to add a comment within your code
morekeywords={*,...}            % if you want to add more keywords to the set
}

\begin{document}

\section{Prerequisites}

You might want to download latest Mapnik release (0.7.1) from here.
There's also a set of geodata files which should be downloaded from here.
For the most adventurous of you, download osm2pgsql from here and the OSM planet file from here.

\section{Math}

\subsection{Projections}

Scale factor
Tranformations of
Areas
Distances
Directions

Tissot indicatrix

graticules


Types of surface:
Cylindrical
Cone
Plane/Azimuthal

Equal area
Equidistant
Conformal
Gnomonic
Compromise

Spatial Reference System


\section{Rendering}

\subsection{Perception}

In case with cartography a careful approach is preferred when choosing
color scheme. Not only color outlines main features of the map, but also
helps achieve the goal of being accessible for everybody.

\subsection{Simple Mapnik rendering}

This is a short example which creates basic map object, applies simple styling
rules and provides a datasource.

\lstinputlisting{simple-rendering.py}

Let's outline several points here.

Creation of map

Creating and applying styles

Attaching datasources

Rendering

\subsection{Simple Mapnik XML}

Of course, defining such logic in code is not only cumbersome, but also makes
such tasks as serialization and changing Mapnik styles on the fly

\lstinputlisting{simple.xml}

IMAGE

Adding Tissot's indicatrix

\lstinputlisting{tissot-indicatrix.xml}

IMAGE

Viewing in different projection

\lstinputlisting{tissot-indicatrix-mercator.xml}

IMAGE

\lstinputlisting{tissot-indicatrix-mercator.xml}

IMAGE



\section{References}

On projections and srs
http://www.sharpgis.net/post/2007/05/Spatial-references2c-coordinate-systems2c-projections2c-datums2c-ellipsoids-e28093-confusing.aspx
http://spatialreference.org/
http://proj.maptools.org/gen_parms.html
http://trac.osgeo.org/proj/wiki/GenParms
http://trac.mapnik.org/wiki/IntroductionToGIS
http://en.wikipedia.org/wiki/Winkel_Tripel_projection



\end{document}
